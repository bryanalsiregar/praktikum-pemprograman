\documentclass{article}
\usepackage{graphicx} % Required for inserting images
\usepackage{hyperref}
\usepackage{minted}

\usepackage[a4paper, margin=1in]{geometry}


\hypersetup{
    colorlinks=true,
    urlcolor=cyan
}

\title{Laporan Praktikum Pemrograman\\Pertemuan 7}
\author{Bryan Al Hilal Siregar\\ 24/541712/PA/22990\\ \href{https://github.com/bryanalsiregar/praktikum-pemprograman}{https://github.com/bryanalsiregar/praktikum-pemprograman}}
\date{\today}

\begin{document}

\maketitle

\section{Latihan}
\begin{enumerate}
    \item Jelaskan perbedaan fungsi dengan nilai balik dan fungsi tanpa nilai balik!
    \item Jelaskan perbedaan $pass$ $by$ $value$ dan $pass$ $by$ $reference$!
\end{enumerate}

\section{Jawaban}
\begin{enumerate}
    \item Fungsi nilai balik adalah fungsi yang memiliki $statement$ $return$, mengembalikan nilai sesuai dengan tipe data fungsi tersebut, dan memiliki ciri khas pada tipe data di samping nama fungsi. Fungsi tanpa nilai balik adalah fungsi yang tidak memiliki $statement$ $return$ sehingga fungsi tidak mengembalikan nilai apa pun dan memiliki ciri khas kata void di samping nama fungsi.
    \item $Pass$ $by$ $value$ adala parameter pada fungsi yang hanya mengambil nilai dan disimpan pada nama variabel pada parameter. $Pass$ $by$ $reference$ adalah parameter pada fungsi yang tidak hanya mengambil nilai, tetapi mengambil alamat dari variabel yang dimasukkan ke dalam fungsi sehingga nilai variabel yang dimasukkan ke dalam fungsi dapat berubah bergantung pada $statement$ pada fungsi. 
\end{enumerate}

\end{document}