\documentclass{article}
\usepackage{graphicx} % Required for inserting images
\usepackage{hyperref}
\usepackage{minted}

\usepackage[a4paper, margin=1in]{geometry}


\hypersetup{
    colorlinks=true,
    urlcolor=cyan
}

\title{Laporan Praktikum Pemrograman\\Pertemuan <N>}
\author{Bryan Al Hilal Siregar\\ 24/541712/PA/22990\\ \href{https://github.com/bryanalsiregar/praktikum-pemprograman}{https://github.com/bryanalsiregar/praktikum-pemprograman}}
\date{<TANGGAL>}

\begin{document}

\maketitle

\section{Cara Menjalankan Program di Codespace}
\begin{enumerate}
    \item Buka meeting\_<n>
    \item Buka assignment\_1 (sesuaikan folder mana yang ingin dilihat dari tugasnya)
    \item Ketik command $make$
    \item Gunakan berbagai macam command $make$ pada catatan sesuai dengan kebutuhan
\end{enumerate}

\section{Catatan}

Untuk menjalankan program dan tes program, jalankan perintah make dengan perintah tambahan sebagai berikut:
\begin{minted}[frame=lines, fontsize=\large]{bash}
    $ make
    $ make run
    $ make test
    $ make clean 
    $ make all
\end{minted}

\begin{enumerate}
    \item \textbf{$make$}: digunakan untuk mencompile file cpp
    \item \textbf{$make$ $run$}: digunakan untuk menjalankan program dengan meminta $input$ dari $user$ dan menampilkan $output$
    \item \textbf{$make$ $test$}: digunakan untuk menjalankan program dengan tes dengan cara kerja membuka $file$ $test.txt$ dan mengambil $input$ dan $output$ yang diharapkan pada $file$ tersebut sesuai format. Terminal akan menampilkan status dari $test$ tersebut
    \item \textbf{$make$ $clean$}: digunakan untuk menghapus nama $file$ yang menjalankan program
    \item \textbf{$make$ $all$}: digunakan untuk men-$compile$ $file$ $coding$-an dengan nama $file$ $compile$ yang sama dengan $file$ $coding$-an
\end{enumerate}

\section{Tugas}

\subsection{<NAMA TUGAS 1>}


\subsection{<NAMA TUGAS 2>}

\section{Penyelesaian}

\subsection{<NAMA TUGAS 1>}

\textbf{Algoritma}

\textbf{Flowchart}

\textbf{Cara Kerja}

\textbf{Output}

\textbf{Output Test}

\subsection{<NAMA TUGAS 2>}

\textbf{Algoritma}

\textbf{Flowchart}

\textbf{Cara Kerja}

\textbf{Output}

\textbf{Output Test}

\end{document}